\documentclass[10pt,a4paper]{article}
\usepackage[a4paper,vmargin={30mm},hmargin={30mm}]{geometry}
\usepackage[utf8x]{inputenc}
\usepackage{ucs}
\usepackage{amsmath}
\usepackage{amsfonts}
\usepackage{amssymb}
\usepackage{parskip} % Skip indentation of first row
\usepackage{graphicx} % Graphics support
\usepackage{longtable} % Tables across several pages
\author{Danilo Bargen}
\title{Theoriesammlung Analysis 2}

\begin{document}

\begin{titlepage}
	\maketitle
	\vspace{120mm}
	\center\includegraphics{hsr_logo.png}
	\thispagestyle{empty} % Don't start page numbers on this page
\end{titlepage}

\tableofcontents\newpage

\section{Integralrechnung}

\subsection{Definition des Integrals}

Die Definition des Integrals lautet
$$\int\limits_a^b f = \lim_{n \mapsto \infty}\left(\sum_{i=1}^n f(x_i) \cdot \Delta x\right)$$
mit
$$\Delta x = \frac{b-a}{n}$$
und
$$x_i = a + i \cdot \Delta x$$


\subsection{Summenformeln}

$$\sum_{i=1}^n i = \frac{n(n+1)}{2}$$
$$\sum_{i=1}^n i^2 = \frac{n(n+1)(2n+1)}{6}$$
$$\sum_{i=1}^n i^3 = \left(\frac{n(n+1)}{2}\right)^2$$


\subsection{Graphische Interpretation von Integralen}

Wir betrachten das Integral
$$\int\limits_a^b f$$
Wir nennen die Fläche, welche horizontal durch zwei Abszissen und vertikal
durch die Abszissenachse und den Funktionsgraphen begrenzt sind als
\textit{Fläche unter dem Funktionsgraphen}. Es sind nun zwei Fälle zu unterscheiden:

\begin{itemize}
\item Wenn $a < b$ ist:\\
    Dann sind Flächen unter Funktionsgraphen mit positiver Ordinate positiv und
    solche mit negativen Ordinaten negativ zu zählen.
\item Wenn $a > b$ ist:\\
    Dann sind Flächen unter Funktionsgraphen mit positiver
    Ordinate negativ und solche mit negativen Ordinaten positiv zu zählen.
\end{itemize}


\subsection{Vorzeichenregeln und Additivität}

Falls die beteiligten Integrale existieren, gilt
\begin{itemize}
\item Vertauschen der Integralgrenzen ändert das Vorzeichen des Integrals
$$\int\limits_a^b f = - \int\limits_b^a f$$
\item Aneinanderstossende Integrale können zusammengefasst werden.
$$\int\limits_a^b f + \int\limits_b^c f = \int\limits_a^c f$$
\end{itemize}


\subsection{Linearitätsregeln für Integrale}

Seien $f$ und $g$ auf dem intervall $[a;b]$ integrierbare Funktionen und $c$ eine
Konstante. Dann gelten die beiden Linearitätsgesetze:
$$\int\limits_a^b (f + g) = \left(\int\limits_a^b f\right) + 
    \left(\int\limits_a^b g\right)$$
$$\int\limits_a^b (c \cdot f) = c \int\limits_a^b f$$


\subsection{Simpson-Regel}

Sei $f$ eine auf $[a;b]$ viermal differenzierbare Funktion und $n$ eine gerade Zahl.
Ferner sei
$$x_i = a + i \cdot \Delta x \textrm{ mit } \Delta x = \frac{b-a}{n}
    \textrm{ und } y_i = f(x_i)$$
Dann ist
$$S_n = \frac{\Delta x}{3}(y_0 + 4y_1 + 2y_2 + 4y_3 + 2y_4 + ... + 4y_{n-1} + y_n)$$
$$= \frac{\Delta x}{3}\left(y_0 + y_n + 4 \sum_{k=1}^{n/2} y_{2k-1} 
    + 2 \sum_{k=1}^{n/2-1} y_{2k}\right)$$
eine Schätzung für das Integral $\int\limits_a^b f$, wobei der Fehler
$$E_n = \left(\int\limits_a^b f\right) - S_n = \frac{b-a}{180}\Delta x^4 f^{(4)}(\xi)
    = \frac{(b-a)^5}{180n^4} f^{(4)}(\xi)$$
$$\textrm{für ein } \xi \in [a;b]$$
beträgt.


\subsection{Integralfunktion}

Gegeben sei eine auf dem Intervall $[a;b]$ integrierbare Funktion $f$. Dann heisst
jede Funktion der Form
$$x \mapsto \int\limits_c^x f$$
für eine Konstante $c \in [a;b]$ eine \textit{Integralfunktion} von $f$.


\subsection{Zusammenhang verschiedener Integralfunktionen}

Verschiedene Integralfunktionen derselben Funktion unterscheiden sich nur durch
eine Konstante. Wenn also
$$\Phi_c := x \mapsto \int\limits_c^x f \textrm{ und } \Phi_d := x \mapsto
    \int\limits_d^x f$$
dann gilt
$$\Phi_d = \Phi_c + k \textrm{ wobei } k \textrm{ konstant}$$



\section{Fouriertransformation}

...


\section{Differenzialgleichungen}

...

\end{document}
